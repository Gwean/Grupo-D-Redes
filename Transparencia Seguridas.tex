
\documentclass{beamer}
\mode<presentation>
{
  \usetheme{Szeged}
 \usecolortheme{Beaver}
  \setbeamercovered{transparent}
}


\usepackage[english]{babel}
\usepackage[utf8]{inputenc}
\usepackage{times}
\usepackage[T1]{fontenc}

\title[Seguridad ]
{Seguridad}

\subtitle
{Grupo D}

\author[Author, Another, Another1] % (optional, use only with lots of authors)
{L.Barletta\inst{1}  \and I. Canut\inst{2} \and G. Mondino\inst{3}}


\institute[I.P.S] % (optional, but mostly needed)
{
  Instituto Politecnico Superior\\
 Gral. San Martin
}
\date
{Conexion de Redes Extendidas, 2019}

\subject{Theoretical Computer Science}


\begin{document}

\begin{frame}
  \titlepage
\end{frame}

\begin{frame}{Contenidos}
  \tableofcontents

\end{frame}

\section{Cifrado}
\subsection{Introduccion a la Criptografia}

\begin{frame}{Cifrado.} 

Algunas definiciones basicas:
  \begin{itemize}
 \item
    \texttt{Criptologia:} Es la suma de la Criptografia y el  Criptoanalisis.  
 \item
    \texttt{Codigo:} Se reemplaza una palabra por otra que significa lo mismo.
\item
    \texttt{Texto Plano:} Es el texto a ser encriptado.
  \item
    \texttt{Texto Cifrado:} Es el texto despues del proceso de encriptacion.
 \item
    \texttt{Intruso Pasivo:} Es capaz de escuchar el trafico, se asume como una posibilidad siempre.
 \item
    \texttt{Intruso Activo:} No solo es capaz de escuchar sino tambien alterar el trafico.
  \end{itemize}

\end{frame}

\begin{frame}{Cifrado} 

    \begin{itemize}
   \item
    \texttt{Cifrado:} Transformacion caracter por caracter sin diferenciar la estructura linguistica del mensaje.  
 \item
    \texttt{Cifrado por Sustitucion:} Tipo de cifrado donde se cambian los valores pero no sus posiciones. 
\item
    \texttt{Cifrado por Transformacion:}  Tipo de cifrado donde se cambian las posiciones de las unidades pero no su valor. 
    \end{itemize}
  
\end{frame}

\end{document}
